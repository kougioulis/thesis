\begin{titlepage}
    \begin{center}
        \Large \textbf{\thesistitle} \\[0.5cm]
        \large \textbf{Abstract} \\[0.5cm]
    \end{center}
    Causal discovery for both cross-sectional and temporal data has traditionally followed a dataset-specific paradigm, where a new model is fitted for each individual dataset. Such an approach underutilizes the potential of multi-dataset and large-scale pretraining, especially given recent advances in foundation models. The concept of \textit{Large Causal Models (LCMs)} envisions a class of pre-trained neural architectures specifically designed for temporal causal discovery. Existing approaches remain largely proofs of concept, typically constrained to small input sizes (e.g., five variables), with performance degrading rapidly to random guessing as the number of variables or model parameters increases. Moreover, current methods rely heavily on synthetic data, generated under arbitrary assumptions, which substantially limits their ability to generalize to realistic or out-of-distribution samples. This work addresses these challenges through novel methods for training on mixtures of synthetic and realistic data collections, enabling both higher input dimensionality and deeper architectures without loss of performance. Extensive experiments demonstrate that LCMs achieve competitive or superior performance compared to classical causal discovery algorithms, while maintaining robustness across diverse domains, especially on non-synthetic data cases. Our findings also highlight promising directions towards integrating interventional samples and domain knowledge, further advancing the development of foundation models for causal discovery.
    \vfill
\end{titlepage}