\begin{titlepage}
    \selectlanguage{greek}
    \begin{center}
        \Large \textbf{\thesistitlegreek} \\[0.5cm]
        \large \textbf{Περίληψη} \\[0.5cm]
    \end{center}
        \small{Η αιτιακή ανακάλυψη (\textlatin{Causal Discovery}), τόσο για συγχρονικά (\textlatin{cross-sectional}) όσο και για χρονικά (\textlatin{temporal}) δεδομένα, ακολούθησε παραδοσιακά ένα πρότυπο ειδικό στο εκάστοτε σύνολο δεδομένων (\textlatin{dataset-specific paradigm}), όπου ένα νέο μοντέλο εκτιμάται για κάθε μεμονωμένο, διαφορετικό σύνολο δεδομένων. Μια τέτοια προσέγγιση υποεκμεταλλεύεται τις δυνατότητες της προεκπαίδευσης πολλαπλών συνόλων δεδομένων και μεγάλης κλίμακας, ειδικά λαμβάνοντας υπ'όψιν των πρόσφατων εξελίξεων στα θεμελιώδη μοντέλα (\textlatin{Foundation Models}). Η έννοια των \textit{Μεγάλων Αιτιακών Μοντέλων (\textlatin{Large Causal Models})} οραματίζεται μια κατηγορία προ-εκπαιδευμένων νευρωνικών αρχιτεκτονικών (Θεμελιώδη Μοντέλα) ειδικά σχεδιασμένων για χρονική αιτιακή ανακάλυψη. Οι υπάρχουσες προσεγγίσεις παραμένουν σε μεγάλο βαθμό απλές επαληθεύσεις ιδεών, περιορισμένες σε μικρά μεγέθη εισόδου (π.χ., πέντε μεταβλητών), με την απόδοση να υποβαθμίζεται γρήγορα σε τυχαία εικασία καθώς αυξάνεται ο αριθμός των μεταβλητών ή των παραμέτρων του μοντέλου. Επιπλέον, οι τρέχουσες μέθοδοι βασίζονται έντονα σε περιορισμένο πλήθος αυθαίρετα παραγόμενων συνθετικών δεδομένων, γεγονός που περιορίζει σημαντικά την ικανότητά τους να γενικεύονται σε ρεαλιστικά ή εκτός κατανομής δεδομένα. Η παρούσα εργασία αντιμετωπίζει τις παραπάνω προκλήσεις μέσω νέων μεθόδων για εκπαίδευση σε μείξεις συνθετικών και ρεαλιστικών συλλογών δεδομένων, επιτρέποντας τόσο υψηλότερη διαστατιμότητα εισόδου όσο και βαθύτερες αρχιτεκτονικές χωρίς απώλεια απόδοσης. Εκτεταμένα πειράματα καταδεικνύουν πως τα Μεγάλα Αιτιακά Μοντέλα επιτυγχάνουν ανταγωνιστική ή ανώτερη απόδοση σε σύγκριση με κλασσικούς αλγόριθμους αιτιακής ανακάλυψης, διατηρώντας παράλληλα στιβαρότητα, ειδικά σε περιπτώσεις μη συνθετικών δεδομένων. Τα ευρήματά μας υπογραμμίζουν επίσης υποσχόμενες κατευθύνσεις προς την ενσωμάτωση παρεμβατικών δειγμάτων και εκ των προτέρω γνώσης, προωθώντας περαιτέρω την ανάπτυξη θεμελιωδών μοντέλων για αιτιακή ανακάλυψη.} \\[0.5cm]
    \selectlanguage{english}
    \vfill
\end{titlepage}