\chapter{\lowercase{d}-Separation}

The central tool connecting conditional independencies of data and their corresponding DAG is the concept of \emph{d-separation} (directional separation) introduced by \citet{pearl1988probabilistic}. We provide its definition for completeness of our writing, as it represents a graphical criterion on whether a set of variables \(\mathbf{X}\) is conditionally independent of another set \(\mathbf{Y}\), given a third set \(\mathbf{Z}\), based on the structure of the DAG. It is defined for the i.i.d. case, but it can also be extended intuit to the time series setting.

\begin{definition}[d-Separation]
Let \(\mathcal{G}\) be a DAG over a set of variables \(\mathbf{V}\). A path \(p\) between nodes \(X\) and \(Y\) is said to be \emph{blocked} by a set of variables \(\mathbf{Z}\) if one of the following holds:
\begin{enumerate}
    \item There exists a chain \(A \rightarrow B \rightarrow C\) or a fork \(A \leftarrow B \rightarrow C\) such that \(B \in \mathbf{Z}\),
    \item There exists a collider \(A \rightarrow B \leftarrow C\) such that \(B \notin \mathbf{Z}\) and no descendant of \(B\) is in \(\mathbf{Z}\).
\end{enumerate}
If all paths between \(X\) and \(Y\) are blocked by \(\mathbf{Z}\), then \(X\) and \(Y\) are said to be \emph{d-separated} given \(\mathbf{Z}\), written \(X \perp\!\!\!\perp Y \mid \mathbf{Z}\).
\end{definition}

D-separation serves as the graphical counterpart to conditional independence in the disentagled factorization \(\mathbb{P}\), under the assumption of the Causal Markov Condition (Section \ref{sec:causal-assumptions}). It plays a fundamental role in constraint-based causal discovery algorithms, such as the PC \citep{spirtes2001causation} algorithm (and importantly, its temporal extension PCMCI \citep{runge2018causal}), which rely on conditional independence tests to infer causal structure.