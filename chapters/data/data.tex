\chapter{Data Generation} \label{chap:data}

\begin{chapquote}{\textit{Abraham Lincoln}}
``If I had six hours to chop down a tree, I'd spend the first four sharpening the axe.''
\end{chapquote}

A central limitation of current causal discovery methods, either in the temporal or i.i.d. setting, is the exclusive reliance on synthetic datasets for their evaluation, benchmarking and recently, for foundational-inspired methods (Section \ref{sec:related-work}), training. This chapter outlines our methodologies for generating vast data instances\footnote{We refer to a pair of a multivariate time-series and its corresponding causal graph as a \textit{data instance} or simply dataset, when no ambiguity exists regarding the form of data. Multiple pairs of data instances are referred to as a \textit{data collection.}} to support robust training of our large causal models. Synthetic data generation pipelines are typically constructed using predefined causal structures and known functional dependencies from randomly generated SCMs. While such approaches offer precise control and clarity over the data generation process, they often fail to capture the complexity and statistical idiosyncrasies of real-world time-series. To address this, we propose generation methods of both \textit{synthetic} and \textit{simulated} (realistically generated) data. Briefly, we opt for training on a mixture of both synthetic and simulated datasets. 

We begin with an introduction to the context of data generation (Section \ref{sec:data-introduction}) and proceed by outlining the main challenges for causal discovery (Section \ref{sec:data-challenge}). The remainder is structured as follows: Section \ref{sec:data-synthetic} outlines the generation process of synthetic data pairs, which serve as a learning foundation for our Large Causal Models. Section \ref{sec:data-simulated} introduces the \textit{Temporal Causal-based Simulation (TCS)} framework, a modular and model-agnostic generative method for causal models and corresponding time-series samples, as well as the \textit{Adversarial Causal Tuning (ACT)} module, which enables fine-tuning of TCS based on a Min-max optimization approach. Unlike traditional synthetic data generators in literature, our methodology learns the structural and functional dependencies, as well as the noise distribution, directly from time-series samples, and by allowing fine-tuning of the process further enhances the realism of generated causal models and bridges the domain gap between synthetic training and applicability to real-world scenarios. This part serves as a crucial component for enhancing the generalization performance of our LCMs, allowing us to leverage data of higher quality for not only training, but also evaluation and benchmarking. Section \ref{sec:data-training} covers the curation of training and validation pairs used for model training, as well as the construction of our benchmark datasets. Finally, Section \ref{subsec:shardification} provides an overview of our approach for partitioning data into shards for efficient training of LCMs.

\section{Introduction} \label{sec:data-introduction}

In the foundation model paradigm, model performance and generalization is closely tied to training data quality and diversity \citep{bommasani2021opportunities}. In the same way foundation models for natural language processing (e.g. Large Language Models-LLMs like the family of GPTs \citep{radford2019language, brown2020language, achiam2023gpt}) and computer vision (e.g. CLIP \citep{radford2021learning}, DALL-E \citep{ramesh2021zero}) rely on vast and heterogeneous data collections to fully capture the complexities of the real world in order to generalize effectively, a foundation model for causal discovery would require training data that reflect the complexities and properties of real-world causal systems.

For the most part, obtaining real data-cases is a straightforward task for domains unrelated to causality. Consider for example the case of constructing a neural model for recognizing images (such as dogs, cats, a glass bottle etc.). Real-world images are widely available online and can even be collected easily nowadays. For a supervised learning like image classification, it is then a matter of \textit{annotating} each image with its ground truth label. Such labeled image collections have already been curated, such as Imagenet \citep{deng2009imagenet}. A similar scenario exists for time-series forecasting models as well \citep{hahn2023time, kim2025comprehensive}. Foundation models for time-series forecasting have been developed and trained on such data collections, such as the Transformer XL \citep{dai2019transformer}, the Informer \citep{zhou2021informer} etc.

An ongoing challenge in causal discovery is obtaining real data-cases, as it represents a much more nuanced task that generating purely synthetic ones. In the absence of real-world data and causal graph pairs, existing works regarding causal discovery on both \textit{iid} and temporal data tend to focus on using either on (i) \textit{purely synthetic} samples (causal models with randomized Erd\H{o}s-Renyi \citep{brouillard2020differentiable} or Barab\'{a}si-Albert \citep{barabasi1999emergence} structures) or (ii) \textit{semi-synthetic}, based on existing knowledge of real cases and scenarios (e.g., simulated Markov processes of physical systems). Examples of such datasets include the Kuramoto model of coupled oscillators \citep{kuramoto1975self, lowe2022amortized} and brain connectivity models such as \(f\)MRI \citep{smith2011network}. Observational data from synthetic causal models are then generated through \textit{ancestral sampling} \citep[Section 4.2.5]{murphy2023probabilistic} from the specified temporal SCM, and interventional data, if needed, in a similar manner. While this approach provides full control over the form of the causal structure and resulting causations, it leads to inherent limitations: synthetic data generation operates under idealized assumptions, including fully specified models, simplified noise distributions, absence of latent confounding, noisy interactions etc. As a result, generated data often fail to capture the complexity, heterogeneity, and non-stationary characteristics of real-world time-series. Real time-series samples are vastly available, but each one must be accompanied with its ground truth causal mechanism, from which data samples are generated from. As many causal assumptions on which causal discovery and causal inference methods are built upon (as elaborated in Chapter \ref{chap:problem-formulation}) are unknown or unlikely to hold in such real-world systems, the need for realistic data cases is even more pressing.

Regarding training of foundation models, the authors of TimesFM \citep{das2024decoder} articulate this principle clearly in the context of time-series forecasting: \\

\begin{chapquote}{\textit{Rajat Sen and Yichen Zhou}, TimesFM authors\footnote{\url{https://research.google/blog/a-decoder-only-foundation-model-for-time-series-forecasting} (Last Accessed: May 2025).}}
``Synthetic data helps with the basics... Real-world data adds real-world flavor... This helps TimesFM understand the bigger picture and generalize better when provided with domain-specific contexts not seen during training.''
\end{chapquote}

This sharp observation holds equally true for foundation models for causal discovery: synthetic data encode fundamental structural and temporal dependencies, while realistic data capture the statistical irregularities and intractable functional complexities inherent in real-world systems.

Surprisingly, causal discovery methods, including recent foundation models presented in Section \ref{sec:related-work} such as CP \citep{stein2024embracing}, SEA \citep{wu2024sample} and Do-PFN \citep{robertson2025do} are trained exclusively on synthetic datasets. Such models are rarely exposed to the structural uncertainty, noise variability or irregular patterns present in real data. Furthermore, their evaluation is typically restricted to synthetic benchmarks, limiting their demonstrated generalization capacity in practical deployment scenarios.

To address the above concerns, our work adopts a dual-strategy approach for data generation. Alongside a conventional yet rigid synthetic data generation pipeline from temporal SCMs, we introduce realistically generated datasets constructed using Adversarial Causal Tuning - ACT, inspired by AutoML \citep{hutter2019automated}. In short, the method generates time-series and corresponding lagged causal graphs by learning directly from real data samples. It decomposes the data generation process into three modular phases: (i) discovery of the causal graph, (ii) approximation of the functional dependencies and (iii) noise distribution modeling, all under a Min-max fine tuning scheme for optimal causal model selection. In this way, we allow generation of time-series that are both causally coherent and statistically realistic.

%To address the above concerns, our work adopts a dual-strategy approach for data generation. Alongside a conventional yet rigid synthetic data generation pipeline from temporal SCMs, we introduce realistically generated datasets constructed using the \textit{Temporal Causal-based Simulation} (TCS) framework. TCS generates time-series and corresponding lagged causal graphs by learning directly from real data samples. In short, it decomposes the data generation process into three modular phases: (i) discovery of the causal graph, (ii) approximation of the functional dependencies and (iii) noise distribution modeling, all under a Min-max fine tuning scheme for optimal causal model selection (\textit{Adversarial Causal Tuning - ACT}) inspired by AutoML \citep{hutter2019automated}. In this way, we allow generation of time-series that are both causally coherent and statistically realistic.

\section{The Challenge} \label{sec:data-challenge}

The task of generating synthetic instances is evidently a more straightforward problem than the one of generating realistic ones. Various works that address synthetic causal data generation have been introduced, in both temporal and atemporal settings \citep{cinquini2021boosting, wen2021causal}, yet few assess the realism and quality of generated data. Concerning time-series, methods taking into account sophisticated and configurable generative frameworks for both data and the causal structure itself exist (such as \cite{lawrence2020cdml}). Methods such as Causal Chambers by \citet{gamella2024causal} for generating realistic time-series data from physical systems under controlled experimental conditions have also been recently introduced. A key difference between our approach and theirs is that our proposed pipeline remains fully algorithmic, enabling scalable and diverse data generation, instead of a closed and controlled experimental setup. The work of \citet{stein2024embracing} also includes a synthetic generator of time-series, as well as the Tigramite package by \cite{runge2019inferring}. A shortcoming of these works is that during ancestral sampling over the causal graph, cuasal effects may grow too large due to the functional relationships used, thus resulting in values outside floating point range. This issue is handled by re-instantiating the sampling process, which can prove time consuming and cumbersome for generating thousands of instances. What's more, some chosen functional relationships or parametric assumptions about the noise distributions may result in causal graphs that are either too trivial for the model to generalize or too hard for the model to extract any meaningful context. Furthermore, the work of \citet{stein2024embracing}, although it follows an additive time-invariant SCM formulation for samples, it directly proceeds by randomly populating the lagged adjacency tensor, which remains of questionable soundness, as no TSCM is directly introduced. Finally, graphs may prove to be too dense (causal graphs should be, in general, sparse), which may lead the model to either overfit or underfit the data. We take into consideration the above points for the introduction of our own synthetic generation pipeline to output coherent time-lagged causal graphs for training and evaluation of our LCMs.

In the case of real-world data, the challenge is much more complex. The main argument against the sole use of synthetic dta is that reliance on data generated under idealized assumptions, e.g., additive noise models, pre-defined structural priors, or linear interactions, limits data quality and as such the generalization capabilities of our LCMs (i.e. Figure \ref{fig:stein_fmri} in Chapter \ref{chap:introduction}). As we elaborate in detail in Section \ref{sec:data-simulated}, common evaluation protocols for realistic data generators use single-metric comparisons \citep{cheng2024causaltime}, which do not fully capture the generative realism or causal fidelity. What's more, assumptions of both purely synthetic and semi-synthetic causally generated data are unlikely to hold in practice, as real-world models are subject to inherent restrictions and do not fit under closed conditions assumed by such methods. Generating datasets that are both causally coherent and statistically realistic remains a core challenge addressed in this work. We call such datasets \textit{simulated} data.

The assumption (stemming from adopting a supervised learning approach as elaborated in Section \ref{sec:problem-formulation}), that \textit{distributions of training and test data cases must highly overlap}, poses another fundamental challenge in data generation. As one may expect, a supervised classification model (not limited to a neural network) trained on linear instances is expected to perform poorly on non-linear cases. This simple observations further highlights the need for \textit{curating a large data collection with various complexities, sizes and causal properties for training and evaluation}. For synthetic cases, this includes a wide family of graph sizes, node degrees, functional relationships and noise distributions, among others. 

To extend the limited dimensionality from five (\(5\)) variables of \citet{stein2024embracing} to more than double (twelve (\(12\))), the need for constructing an optimal curation of both synthetic and simulated data remains even more significant. The challenge of constructing a vast collection of data (among other number of variables, lags of causal relationships, functional dependencies etc.) is evident in order to avoid degradation of model performance as input dimensionality and model parameters increase (Figure \ref{fig:stein_auc} in Chapter \ref{chap:introduction}).

Finally, one should be accounting for causal assumptions that are not only reflected in the training data but also in the assumptions of the model. For instance, consider generating observational samples from a causal structure \(X_{t} \leftarrow Z_{t-1} \rightarrow Y_{t}\) where \(Z\) is assumed to be a latent confounder and then removed from the corresponding SCM. A model that assumes causal sufficiency would potentially infer (based on these observational samples) a directed edge between \(X_t\) and \(Y_{t}\) (of any lag), which clearly is not the case and results in false interpretation of discovered causal relationships.

\section{Generating Synthetic Causal Models} \label{sec:data-synthetic}

Recall from Chapter \ref{chap:problem-formulation} that we formulate causal mechanisms as a (temporal) SCM, represented by a tuple \((\mathcal{G}, \mathcal{F}, \mathcal{E})\), where \(\mathcal{G}\) is the causal DAG (TSCM), \(\mathcal{F}\) the set of functional dependencies, and \(\mathcal{E}\) the set of noise distributions. Loosely speaking, a generator of random TSCMs and observational samples thus consists of a two-step process. First, a causal DAG instance \(\mathcal{G}\) is sampled from a predefined family of DAGs \(\mathbb{G}\) with various random properties. These may include the number of nodes, edge density (e.g., in-degree and out-degree distributions), the maximum number of lags \(\ell_\text{max}\) in the case of temporal SCMs, and the overall graph structure (e.g., uniform distribution of degrees). With the causal structure known, functional dependencies \( f_i = f(V^i_t | \text{Pa}(V^i_t))\) are sampled for each \(X_i \in \mathcal{G}\) given its parents in the DAG, from known functional families \(\mathcal{F}_1, \ldots,\mathcal{F}_k\), along with noise distributions \( \epsilon_i \sim \mathcal{E}_1, \ldots, \mathcal{E}_m\). For nodes \(V^i_t\) where \(\text{Pa}(V^i_t) = \emptyset\), values are sampled directly from the assumed noise distribution. With both the causal structure and quantitative dependencies defined, the second step consists of generating samples from the SCM.

To illustrate sample generation, consider a simple i.i.d. example with three variables \(X, Y, Z\) and the causal DAG described by the collider \(X \rightarrow Z \leftarrow Y\). This causal graph is associated with the functional dependency \(Z := f(X,Y,\epsilon_Z)\) and noise distributions \(X \sim \epsilon_X\), \(Y \sim \epsilon_Y\). Observational samples are obtained by respecting the topological order of the DAG and generating samples recursively over the descendants (similar to ancestral sampling in latent variable models \citep{murphy2023probabilistic}). For instance, one sample may be \(X = 0.02, Y=0.12\) and \(Z = f(0.02,0.12,0.03) = 1\), where values from the noise distributions are substituted into the functional dependencies.

The topology of each synthetic SCM depends on the random graph generation method. Two main methods exist for random DAG generation in the literature. The first follows the \textit{Erd\H{o}s-Renyi (ER)} paradigm \citep{fienberg2012brief}, as in \citet{hagberg2008exploring}, with variations also used by \citet{brouillard2020differentiable} and \citet{ke2022learning}. Generation begins with an empty graph for a given number of nodes, then randomly and equiprobably picks directed edges to add based on a probability \(p\). Unlike \citet{hagberg2008exploring}, acyclicity checks are applied to reject edges that would introduce cycles, as shown in Algorithm \ref{alg:erdos-renyi}. This process produces approximately uniform node degree distributions.

The second approach, the \textit{Barab{\'a}si-Albert (BA)} algorithm \citep{hagberg2008exploring,barabasi1999emergence}, generates scale-free graphs whose degree distribution follows a power law. Such networks exhibit a “rich-get-richer” phenomenon where a few nodes accumulate disproportionately many connections, a property often seen in real-world networks like social graphs.

In our experiments, we adopt the \textit{Erd\H{o}s-Renyi} method for generating \textit{random lagged temporal causal graphs} (Figure \ref{fig:temporal-graphs} (a)), as it allows fine-grained control over edge probabilities and density. In time-series settings, the additional temporal dimension must be considered. Since we are interested in lagged causal graphs, Algorithm \ref{alg:erdos-renyi} is adapted as follows: the number of nodes \(V\) is sampled from a discrete uniform distribution \(\mathcal{U}(V_\text{min}, V_\text{max})\), the edge probability \(p\) from \(\mathcal{U}(0, 1)\), and the maximum causal effect lag \(\ell_\text{max}\) from \(\mathcal{U}(1, \ell_\text{max})\). The resulting time-lagged graph contains \(V \cdot \ell_\text{max}\) lagged nodes at timesteps \(\tau=1, \dots, \ell_\text{max}\) and \(V\) nodes at the current timestep \(\tau=t\). Directed edges (heading forward through time) are added randomly according to the chosen \(p\), ensuring acyclicity. Furthermore, \textit{no contemporaneous edges are assumed}, i.e., edges of the form \(V^i_l - V^j_l\) do not exist.

The \textit{Barab{\'a}si-Albert} model, while useful for studying scale-free networks, tends to create overly connected hubs that are less representative of causal graphs in scientific domains, and is therefore left unconsidered.

 \begin{algorithm}
 \caption{Erd\H{o}s-R\'{e}nyi DAG generator} \label{alg:erdos-renyi}
 \begin{algorithmic} [1]
 \renewcommand{\algorithmicrequire}{\textbf{Input:}}
 \renewcommand{\algorithmicensure}{\textbf{Output:}}
 \REQUIRE $n$, $p$
 \ENSURE  a DAG $\mathcal{G}$
  \STATE initialize $\mathcal{G}$, $\mathcal{G}^{'}$ as empty directed graphs with $n$ nodes
  \STATE initialize $edges$ with all the possible directed arcs in $\mathcal{G}$, $m=n \cdot p$
\WHILE {$|edges|>0$ and (number of edges in $\mathcal{G} \le m$)}
    \STATE $e \leftarrow \text{pop(shuffle(edges))}$
    \IF {$p \le$ random uniform sample in $[0,1]$}
        \STATE $D \leftarrow \mathcal{G}^{'}$
        \STATE add $e$ to $D$
        \IF {$V$ has a cycle}
            \STATE $D \leftarrow \mathcal{G}^{'}$
        \ELSE 
            \STATE add $e$ to $\mathcal{G}$
            \STATE add $e$ to $\mathcal{G}^{'}$
        \ENDIF
    \ENDIF
\ENDWHILE
\RETURN $\mathcal{G}$ 
\end{algorithmic}
\end{algorithm}

After generating the DAG structure with the desired properties, functional dependencies \(f^j (\text{Pa}( V^j_t), \epsilon^j_t)\) for each variable \(V^j_t \in \mathcal{G}\) are sampled from known functional families \(\mathcal{F}_1, \ldots, \mathcal{F}_k\), along with the corresponding noise distributions. This finalizes the SCM, from which observational data can be sampled.

Generation of samples is carried out using causal ancestral sampling, similar to ancestral sampling in latent variable models (LVMs) \citep{murphy2023probabilistic}. In short, variables without parents are sampled from their noise distributions, and variables with parents are computed from their functional dependencies, respecting topological order. In the temporal case, parent variables are fetched from lagged causal parents, and initial “warm-up” samples are discarded to ensure stability in the generated time-series, following \citet{runge2018causal}. Assuming a known time-lagged SCM \(\mathcal{G}\) with maximum lag \(\ell_{\text{max}}\), the process is described in Algorithm \ref{alg:scm-ancestral-sampling}.

\begin{algorithm}[ht!]
\caption{Temporal SCM Ancestral Sampling (ANCESTRAL)}
\label{alg:scm-ancestral-sampling}
\begin{algorithmic}[1]
\renewcommand{\algorithmicrequire}{\textbf{Input:}}
\renewcommand{\algorithmicensure}{\textbf{Output:}}
\REQUIRE{Temporal Causal Graph \( \mathcal{G} \), max lag \( \ell_\text{max} \), number of timesteps \( T \), number of warmup steps \( W \)}
\ENSURE Generated time-series $\{\mathbf{X}_t\}_{t=1}^{T}$
\STATE Initialize $X^i_{t=0}$ for all $i \in \{1, \dots, N\}$ with random noise
\FOR{$t = W + \ell_{\text{max}} + 1$ to $T$}
    \FOR{each variable $X^i_t$ in topological order of $\mathcal{G}$}
        \STATE Determine lagged parents $\mathrm{Pa}_{X^i_t} \gets \{X^j_{t-k} \mid (X^j, X^i) \in \mathcal{G}, 1 \leq k \leq \ell_{\text{max}}\}$
        \STATE Compute $X^i_t \gets f_i(\mathrm{Pa}_{X^i_t}) + \epsilon^i_t$
    \ENDFOR
\ENDFOR
\RETURN $\{\mathbf{X}^i_t\}_{W+\ell_{\text{max}}}^{T}$
\end{algorithmic}
\end{algorithm}

We now elaborate on the generation hyperparameters used for synthetic sample creation. We select \(V_\text{min}=3, V_\text{max}=12\) and \(\ell_{\text{max}} = 3\). Two-variable SCMs are trivial under our assumptions (edges move forward in time and no latent confounders exist). The choice \(V_{\max}=12\) more than doubles the dimensionality of current state-of-the-art datasets, while \(\ell_{\max}=3\) balances (i) realism in short-term dependencies and (ii) tractability in training. Sampling graphs up to these limits aligns with the design constraints of our LCMs (as discussed in Section \ref{sec:problem-formulation} and Chapter \ref{chap:architecture}). A summary of these parameters is shown in Table \ref{tab:scm-parameters}.

\begin{table}[h]
\centering
\caption{Summary of synthetic SCM generation parameters.} \label{tab:scm-parameters}
\begin{tabular}{lllp{5cm}}
\toprule
\textbf{Parameter} & \textbf{Range / Values} & \textbf{Sampling}\\
\midrule
\# Variables \( N \) & 3 to 12 & Uniform discrete \\
\# Lags \( \ell_{\max} \) & 1 to 3 & Uniform discrete \\
Functional dependencies & See Table \ref{tab:functions} & Uniform discrete \\
Noise distributions & See Table \ref{tab:noise-dists} & Categorical \\
Edge probabilities \( p_{\text{edge}} \) & Dynamic & Categorical \\
\bottomrule
\end{tabular}
\end{table}

To prevent pathological graph densities, \(p_\text{edge}\) is scaled adaptively to the number of possible lagged edges \(E=N^2 \cdot \ell_{\max}\). Specifically, \(p_\text{edge}\) is sampled from a discrete set of values inversely proportional to \(E\), with categorical weights favoring sparsity (Table \ref{tab:p-edge-ranges}). If no edges are sampled during graph generation, \(p_\text{edge}\) is iteratively increased by \(1\%\) until a valid graph is obtained. This dynamic adjustment ensures diversity across sparse, medium, and dense regimes while maintaining realism. Additionally, \(p_\text{edge}\) is upper bounded by \(0.15\) to avoid dense graphs.

\begin{table}[h]
\centering
\caption{Functional dependencies used in synthetic SCM generation.} \label{tab:functions}
\begin{tabular}{lll}
\toprule
\textbf{Family} & \textbf{Functional form} & \textbf{Description} \\
\midrule
Linear & \(f(x) = a x + b\); & Additive linear effect \\
       & \(f(x_1, x_2) = a_1 x_1 + a_2 x_2 + b\) & Linear combination of parents \\
\midrule
Non-linear & \(f(x) = \alpha_nx^n + \ldots \alpha_0x_0 \); & Polynomial \\
           & \(f(x) = e^x \); & Exponential Transformation \\
           & \(f(x) = \sin(x)\); & Sinusoidal \\
           & \(f(x_1, x_2) = x_1 x_2\); & Multiplicative interaction \\
           & \(f(x) = \log(1 + |x|)\); & Logarithmic transformation \\
\midrule
Bounded & \( f(x) = \tanh(x) \) & Hyperbolic tangent (bounded in \([-1,1]\)) \\
        & \( f(x) = \sigma(x) \) & Sigmoid \( \big(\frac{1}{1+e^{-x}}\big) \) \\
\bottomrule
\end{tabular}
\end{table}

\begin{table}[h]
\centering
\caption{Noise distributions used in synthetic SCMs.} \label{tab:noise-dists}
\begin{tabular}{lll}
\toprule
\textbf{Name} & \textbf{Distribution} & \textbf{Parameters} \\
\midrule
Gaussian & \( \mathcal{N}(\mu, \sigma^2) \) & \(\mu=0, \sigma=0.25\)\\
Uniform & \( \mathcal{U}(a, b) \) & \( a=0, b=1 \) \\
\bottomrule
\end{tabular}
\end{table}

Functional forms include linear, polynomial, multiplicative, bounded, and sinusoidal mappings, selected from Table \ref{tab:functions}. Each causal edge is assigned a strength coefficient \(\alpha_{ji\lambda} \sim \mathcal{U}(0.3, 0.5)\), controlling the contribution of each parent variable. These coefficients populate the nonzero entries of the lagged adjacency tensor \(\mathbb{A} \in \mathbb{R}^{V \times V \times \ell_{\text{max}}}\), interpreted as edge-level causal strengths rather than causal effect estimates. Noise distributions are sampled from those listed in Table \ref{tab:noise-dists}.

Finally, each synthetic sample is padded or truncated to a fixed sequence length \(L_\text{max}=500\) and variable dimension \(V_\text{max}=12\) to ensure batch processing and model input compatibility (see Section \ref{sec:input-handling}).

\begin{table}[h]
\centering
\caption{Edge probability values used for different synthetic causal graph densities. \(E\) denotes the number of possible lagged edges.} \label{tab:p-edge-ranges}
\begin{tabular}{ccccl}
\toprule
\textbf{Edge Regime} & \textbf{Threshold} & \textbf{Values} & \textbf{Weights} & \textbf{Notes} \\
\midrule
Small graphs & \( E < 100 \) & \( \frac{3}{E}, \frac{5}{E}, \frac{7}{E} \) & \( 0.6, 0.3, 0.1 \) & Promotes sparsity \\
Medium graphs & \( 100 \le E < 200 \) & \( \frac{5}{E}, \frac{7}{E}, \frac{9}{E} \) & \( 0.6, 0.3, 0.1 \) & Moderate edge density \\
Large graphs & \( E \ge 200 \) & \( \frac{9}{E}, \frac{12}{E}, \frac{15}{E} \) & \( 0.6, 0.3, 0.1 \) & Higher density permitted \\
\bottomrule
\end{tabular}
\end{table}

\section{Generating Simulated  Pairs} \label{sec:data-simulated}

In contrast to synthetic data, simulated (realistically generated) data aim to reflect the underlying causal dynamics of real-world systems by reverse-engineering the structural and functional dependencies from actual time-series. Essentially, we address the problem of generating time-series data from a causal model with the same distribution as a given real dataset, that is, constructing a probabilistic causal digital twin. 

For this purpose, we introduce the \textit{Temporal Causal-based Simulation (TCS)} framework, a modular and model-agnostic generative \footnote{A generative model is a type of machine learning model designed to learn the underlying probability distribution of a dataset and is therefore capable of generating new, synthetic data samples that are statistically and structurally similar to the training data.} method for generating time-series along with their temporal causal graphs, which incorporates a fine-tuning process for optimal model selection called \textit{Adversarial Causal Tuning (ACT)}. Generated data is not only statistically coherent but also maintains interpretable causal consistency, making it suitable for training and benchmarking of our Large Causal Models.

Ideally, what one is looking for is the ability to generate data that is both realistic and comparable to real-world cases, and as such a way to discrimimate between simulated data of good and poor quality. In literature, there exist various methods for asserting whether two data samples resemble each other. One way to measure similarity between (observed) real and synthetic data is by the \textit{Maximum Mean Discrepancy (MMD)} \citep{gretton2006kernel} metric. MMD measures the distance between two probability distributions by embedding the data into a reproducing kernel inner-product space and calculating the difference in the mean embeddings. The MMD score is given by \(\text{MMD}^2(X, Y) = \mathbb{E}_{X, X}[k(X, X)] + \mathbb{E}_{Y, Y}[k(Y, Y)] - 2\mathbb{E}_{X, Y}[k(X, Y)]\), where \(X\) and \(Y\) are the real and synthetic data samples respectively and the expectations are taken over all points in the selected data batch. 

A more useful way to assess data similarity is using \textit{Classifier 2-sample tests (C2STs)}. The main idea behind C2STs, introduced by \citet{gretton2012kernel}, is to train a machine learning classifier \(f_\theta\) to distinguish samples \(X=\left\{x_1,\ldots,x_n\right\} \) sampled from distribution \(P\) against samples \(Y = \left\{y_1,\ldots,y_m\right\}\) sampled from distribution \(Q\). Intuitively, if the null hypothesis \(H_0: P = Q\) holds, then the classifier (trained on the labeled training set \(\mathcal{D} = \left\{ (x_i, 0)\right\} \cup \left\{ (y_j, 1)\right\}\)) should be unable to distinguish between samples and should demonstrate performance close to \(1/2\), or chance level when evaluated on a holdout test set. By performance, we refer to the AUC-ROC score of the classifier \citep{bradley1997use}, to which we refer to as \textit{discrimimation score} \(\mathrm{AUC}_D\). As \citet{lopez2017revisiting} demonstrated, classifier 2-sample tests offer several attractive properties for evaluating data similarity, such as assessing the quality of samples generated by models where the likelihood is intractable (cannot be computed analytically) but sampling from the generating process is tractable.

To highlight our contributions in realistic data generation, we present key arguments against the suitability of the work by \citet{cheng2024causaltime} for training and evaluating our LCMs. It remains the closest to our work, in the sense that it is applicable to time-series and has been introduced as a method to benchmark causal discovery algorithms. However, it faces several shortcomings. It begins by fitting a residual-based neural network (\textit{causally disentangled neural network - CDNN}), based on either an MLP or an LSTM architecture, to model a \textit{vector autoRegressive (VAR)} process, using all historical variables for a specified maximum lag; it then extracts a \textit{hypothetical causal graph (HCG)} based on computed feature importance values, using methods such as deep Shapley additive exPlanations (DeepSHAP) \citep{lundberg2017unified}, while being able to generate data auto-regressively from the fitted model. As the authors themselves point out throughout their paper, the extracted HCG using feature importance does not constitute causal discovery and does not represent a causal model. As such, causal queries are not possible and the discovered graph does not represent a causal digital twin. As their extracted hypothetical causal graph represents a \textit{summary graph} with no lag information and not a \textit{lagged causal graph} (Figure \ref{fig:temporal-graphs}), its scope remains limited. Additionally, although a generative mechanism for data exists by the fitted model, it does not reflect or approximate the underlying TSCM mechanism. Consequently, generation of interventional samples is unfeasible: the resulting process may generate temporally correlated samples, but not reason causally, e.g. to estimate \(P(Y|\text{do}(X=x))\). 

\begin{figure}[ht!]
    \centering
    \begin{minipage}[t]{0.32\textwidth}
        \centering
        \includegraphics[width=0.99\linewidth]{images/figures/t-a.png}
        \label{fig:tsne-a}
    \end{minipage}%
    \hfill
    \begin{minipage}[t]{0.32\textwidth}
        \centering
        \includegraphics[width=0.99\linewidth]{images/figures/t-b.png}
        \label{fig:tsne-b}
    \end{minipage}%
    \hfill
    \begin{minipage}[t]{0.32\textwidth}
        \centering
        \includegraphics[width=0.99\linewidth]{images/figures/t-c.png}
        \label{fig:tsne-c}
    \end{minipage}

    \vspace{15pt}

    \caption{t-SNE \citep{maaten2008visualizing} projections of original and simulated data across different datasets. Optimized C2ST variants outperform standard methods (e.g., MMD) in detecting distributional differences, even when visual separability is low.}
    \label{fig:tsne}

    \vspace{15pt}
\end{figure}

Furthermore, CausalTime relies on a single discriminator to evaluate its simulation quality in addition to the MMD metric. As shown in Figure \ref{fig:tsne} leads to misleading conclusions about the quality of generated data. Data are projected in the plane using t-SNE \citep{maaten2008visualizing}, and the discriminator is trained to distinguish between real and generated data. In Figure \ref{fig:tsne} (a), only the support vector classifier (SVC)-based C2ST is able to distinguish simulated from original samples, which have been generated using trigonometric functions. The other two C2STs, based on logistic regression (LR) and long-short-term memory (LSTM) are unable to do so, with an \(\mathrm{AUC}_D\) of \(0.52\) and \(0.61\) respectively. A similar experimental scenario is shown in Figure \ref{fig:tsne} (b) on the \texttt{AirQualityUCI} dataset where an optimized\footnote{An optimized classifier refers to a model whose hyperparameters or configuration are tuned to maximize its performance (e.g., \(\mathrm{AUC}_D\)), whereas an unoptimized classifier uses default or untuned parameters.} LSTM classifier discriminates, while the unoptimized one fails. Finally, Figure \ref{fig:tsne} (c) reveals the inadequacy of the maximum mean discrepancy metric, as although it is close to zero and data seem indistinguishable with the naked eye, an optimized LSTM classifier is still able to separate them. Consequently, the need for introducing a causal model generator by utilizing optimized C2STs is clear.

\subsection{Temporal Causal-based Simulation (TCS)} \label{subsec:tcs}

This subsection is dedicated to the main generative method for producing realistic time-series data and their respective time-lagged causal graph, called \textit{Temporal Causal-based Simulation (TCS)}, depicted in Figure \ref{fig:pipeline-part-a}. Our method can be summarized briefly by the following statement: \textit{Select the model that generates simulated data which, despite our best efforts, are indistinguishable from the real training data}. By best efforts, we refer to the fine-tuning phase. We summarize the formulation as follows: (i) a \textit{Generation phase}, where a module performs a search in the space of causal models by generating TSCMs and (ii) a \textit{Tuning phase}, where another module searches the space of discriminators to distinguish between real and simulated data, called \textit{Adversarial Causal Tuning (ACT)}. The above procedure leads to a Min-max selection scheme, similarly to generative adversarial networks (GANs) \citep{goodfellow2014generative}. We now describe the main phases of the TSCM search space and ACT.

\begin{figure}[tbp]
    \centering
    \includegraphics[width=\textwidth]{images/figures/pipeline_aaa.png}
    \caption{
        Given a multivariate time-series sample of real data, the method begins by estimating the lagged causal graph through temporal causal discovery (\textit{Phase 1}). It then estimates the functional dependencies between variables using a wide range of forecasters (\textit{Phase 2}). Based on the predictions of these forecasters, it learns the noise distribution through a variety of density estimation methods (\textit{Phase 3}). Each combination of methods for Phases 1, 2 and 3 constitutes a configuration for TCS, that results in a unique \textit{temporal causal model}. By considering different configurations, TCS creates a search space for the temporal causal model that best describes the real data.
    }
    \label{fig:pipeline-part-a}
\end{figure}

\subsubsection{Generation of TSCMs}

\paragraph{Phase 1: Causal Discovery}

Initially, TCS retrieves the data's causal structure. To achieve this, the method employs a variety of causal discovery methods for temporal data, either established as a gold standard or express the current state-of-the-art. Specifically, we consider the algorithms \textit{PCMCI} \citep{runge2018causal} and \textit{DYNOTEARS} \citep{pamfil2020dynotears}. PCMCI is a CD method based on conditional independence testing, serving as a temporal extension of PC \citep{spirtes2001causation} established through literature, while DYNOTEARS constitutes an extension of the NOTEARS \citep{zheng2018dags} CD algorithm on temporal data, based on continuous optimization. These methods are also used as benchmarks for evaluating our trained LCMs and are discussed in detail at Section \ref{sec:baselines}.

We now describe the generation of TSCMs with some mathematical rigor. Given a multivariate time-series dataset \( \mathcal{D} \) over \( \mathbf{V} \), the first phase performs temporal causal discovery using an CD algorithm dependent on a given parameter configuration \( \mathbf{B}_\text{CD} \) (e.g. for DYNOTEARS, the regularization constants \( \lambda_\mathbf{W}, \lambda_\mathbf{A} \in \mathbf{B}_\text{CD} \)). That is, obtain: 

\begin{equation}\label{eq:phase-cd}
\mathcal{G}, \left\{ \mathbf{Pa}_{X^i} \right\}_{i \in \mathbf{V}} \leftarrow \text{CAUSAL\_ALG}(\mathcal{D}, \mathbf{B}_\text{CD})
\end{equation}

It is crucial to highlight that the outcome of all the considered time-series CD methods are \textit{lagged causal graphs}, that not only serve as an expressive representation of the causal model for training our LCMs, but also able to preserve the temporal structure of data for further use. We describe TCS in Algorithm \ref{alg:tcs}. It should also be noted that as causal discovery algorithms are governed by a set of causal assumptions, sampled data from TCS, as well as input data to the method, are assumed to follow these assumptions. The algorithms and their assumptions are selected to align with the causal assumptions of our LCMs, in Chapter \ref{chap:architecture}. 

\begin{algorithm}[t]
\caption{Temporal Causal-Based Simulation (TCS-SIMULATE)}
\label{alg:tcs}
\begin{algorithmic}[1]
\renewcommand{\algorithmicrequire}{\textbf{Input:}}
\renewcommand{\algorithmicensure}{\textbf{Output:}}
\REQUIRE Real time-series dataset $\mathcal{D} = \{\mathbf{X}_t\}_{t=1}^{T}$ over variable set $\mathbf{V}$; sample size $n$; causal discovery configs $\{\mathbf{B}_{CD}\}$; functional dependency estimator configs $\{\mathbf{B}_{pred}\}$; noise estimation configs $\{\mathbf{B}_{noise}\}$; ACT parameters $(D, \{C_d\}, \alpha, n_{perm})$
\ENSURE Optimal causal model $s_{\text{best}}$, generated dataset $\mathcal{D}'_{s_{\text{best}}}$, and discriminator score $\mathrm{AUC}_{\text{best}}$
\STATE $\mathcal{S} \gets \text{GENERATE\_TSCM\_CONFIG\_SPACE}(\{\mathbf{B}_{CD}\}, \{\mathbf{B}_{pred}\}, \{\mathbf{B}_{noise}\})$
\STATE $\mathcal{R} \gets \emptyset$
\FOR{each configuration $s \in \mathcal{S}$}
    \STATE Obtain $\mathbf{b}_{CD}$, $\mathbf{b}_{pred}$, $\mathbf{b}_{noise}$ from $s$
    \STATE $(\mathcal{G}_s, \mathbf{Pa}_s) \gets \text{CAUSAL\_ALG}(\mathcal{D}, \mathbf{b}_{CD})$ \COMMENT{Phase 1: Causal discovery}
    \STATE Initialize $\mathcal{F}_s \gets \emptyset$, $\mathcal{E}_s \gets \emptyset$
    \FOR{each variable $X^i \in \mathbf{V}$}
        \IF{$\mathbf{Pa}_{X^i} \neq \emptyset$}
            \STATE $f_{X^i} \gets \text{FIT}(X^i, \mathbf{Pa}_{X^i}, \mathbf{b}_{pred})$
            \STATE $R_{X^i} \gets X^i - f_{X^i}(\mathbf{Pa}_{X^i})$
        \ELSE
            \STATE $R_{X^i} \gets X^i - \mathbb{E}[X^i]$
        \ENDIF
        \STATE $\mathcal{F}_s \gets \mathcal{F}_s \cup \{f_{X^i}\}$
    \ENDFOR
    \FOR{each variable $X^i \in \mathbf{V}$}
        \STATE $\epsilon^i \gets \text{FIT}(R_{X^i}, \mathbf{b}_{noise})$
        \STATE $\mathcal{E}_s \gets \mathcal{E}_s \cup \{\epsilon^i\}$
    \ENDFOR
    \STATE $\mathcal{D}'_s \gets \text{ANCESTRAL}(\mathcal{G}_s, \mathcal{F}_s, \mathcal{E}_s, n, T)$ \COMMENT{Ancestral sampling}
    \STATE $\mathcal{R} \gets \mathcal{R} \cup \{(s, \mathcal{G}_s, \mathcal{F}_s, \mathcal{E}_s, \mathcal{D}'_s)\}$
\ENDFOR
\STATE $(s_{\text{best}}, \mathrm{AUC}_{\text{best}}) \gets \text{ACT}(\mathcal{S}, \mathcal{D}, D, \{C_d\}, \alpha, n_{perm})$
\RETURN $s_{\text{best}}, \mathcal{D}'_{s_{\text{best}}}, \mathrm{AUC}_{\text{best}}$
\end{algorithmic}
\end{algorithm}


\paragraph{Phase 2: Estimation of Functional Dependencies} 

After the causal structure of the data has been retrieved, we estimate the functional dependencies between variables by fitting forecasting methods on the past (lagged) parent values of each variable. We incorporate three model options for the forecasting task. The first consists of classic \textit{Random Forest} (RF) \citep{breiman2001random} models on lagged data representations according to the maximum discovered time lag \(\ell_\text{max}\) during the fist phase of the algorithm. The second option incorporates the \textit{AD-DSTCN}, a 1-D Convolutional forecaster utilized by \citet{nauta2019causal} in their proposed TCDF method. The last addition to our forecaster options is the TimesFM foundation model for time-series forecasting, introduced by \citet{das2024decoder}. Considering the time-series \(X^i\), we now apply (or fit) a predictive model \(f\) on its lagged parents \(\text{Pa}_{X^i}\), to obtain an approximation of the functional dependency as in the SCM formulation. Mathematically, 

\begin{equation}\label{eq:phase-fr}
f(X^i \mid \text{Pa}_{X^i}) \gets \text{FIT}(X^i, \text{Pa}_{X^i}, \mathbf{B}_\text{pred})
\end{equation}

where \(\text{FIT}\) corresponds to the predictive algorithm used. Like in the causal discovery phase, \(\mathbf{B}_\text{pred}\) is the algorithm-dependent parameter configuration. We depict the set of the fitted functional dependencies for each node as \(\mathcal{F}\).

\begin{table}[t]
\centering
\caption{Search spaces for TSCMs and Discriminators (C2STs).} \label{tab:search-spaces}
\scriptsize
\renewcommand{\arraystretch}{1.1}
\setlength{\tabcolsep}{5pt}

\begin{tabular}{@{}lll@{}}
\toprule
\textbf{Method} & \textbf{Hparam} & \textbf{Values} \\
\midrule
\multicolumn{3}{c}{\textbf{(a) TSCM Search Space}} \\
\midrule
\multirow{2}{*}{PCMCI \citep{runge2018causal}}
    & n\_lags & \{2, 3\} \\
    & n\_reps & \{10\} \\

\midrule
\multirow{6}{*}{DYNOTEARS \citep{pamfil2020dynotears}}
    & n\_lags & \{1, 3\} \\
    & $\lambda_w$, $\lambda_a$ & \{0.1\} \\
    & max\_iter & \{100\} \\
    & n\_reps & \{10\} \\
    & Thresholded / Threshold & \{True / 0.05\} \\

\midrule
DENSE (FC Baseline) & Graph Type & Fully Connected \\

\midrule
\multirow{1}{*}{Random Forest \citep{breiman2001random}}
    & n\_estimators & \{100, 500, 1000\} \\

\midrule
\multirow{4}{*}{AD-DSTCN (TCDF) \citep{nauta2019causal}}
    & num\_levels & \{0, 2\} \\
    & epochs & \{1000\} \\
    & kernel\_size / dilation\_c & \{(2,2), (3,3)\} \\
    & lr & \{0.01, 0.001\} \\

\midrule
TimesFM \citep{das2024decoder} & Model Type & Foundational Forecaster \\

\midrule
Noise Models & noise\_approx. & \{\texttt{est}, \texttt{normal}, \texttt{uniform}, \texttt{spline}, \texttt{nvp}\} \\

\midrule
\multicolumn{3}{c}{\textbf{(b) Discriminator (C2ST) Search Space}} \\
\midrule

\multirow{4}{*}{SVM}
    & C & \{1.0, 0.75, 0.5, 0.25\} \\
    & Kernel & \{\texttt{linear}, \texttt{poly}, \texttt{rbf}\} \\
    & Degree & \{3\} \\
    & Gamma & \{\texttt{auto}, \texttt{scale}\} \\

\midrule
\multirow{7}{*}{LSTM}
    & Batch Size & \{32, 64\} \\
    & Hidden Size & \{128, 256\} \\
    & Layers & \{2, 3\} \\
    & Dropout & \{0.05, 0.1\} \\
    & Seq. Len & \{10, 20\} \\
    & Epochs & \{10, 50\} \\
    & LR & \{0.0001, 0.001\} \\

\bottomrule
\end{tabular}
\end{table}

\paragraph{Phase 3: Estimation of Noise Distribution}

The estimation of functional dependencies is followed by the estimation of the true noise distribution for each variable. The predictions of the fitted forecasting models are utilized to obtain the residual terms by subtracting the real data values of the corresponding time-steps. In case a variable at hand has no deduced causal parents from Phase 1, a trivial predictor can be used, as the mean of the observed values, a solution dependent on the assumption of time-series stationarity. The result of this process corresponds to the empirical distribution of the noise, which is then utilized along additional noise estimation methods. That is, for a variable \( X_i \) and a noise configuration \(\mathbf{B}_\text{noise}\), fit the noise estimator \(\epsilon_i \leftarrow \text{FIT}(\mathbf{B}_\text{noise})\).

Additionally, we also consider two neural approaches for the estimation of the noise distribution. These consists of Neural Spline Flows \citep{durkan2019neural} and a simplistic version of the RealNVP model \citep{dinh2016density}. Finally, a known parametric distribution for the noise of the data is utilized (e.g., Normal) and the empirical noise distribution is used to fit its parameters (e.g., mean and variance). The set of all fitted noise estimators for each node is depicted as \(\mathcal{E}\).

Table \ref{tab:search-spaces} provides an overview of the implemented methods for each phase, along with their hyperparameter configurations. Their successive application results in the creation of a TSCM \((\mathcal{G}, \mathcal{F}, \mathcal{E})\), consisting of the discovered causal structure \( \mathcal{G}\) (\textit{phase 1}), functional dependencies \( f^i \in \mathcal{F}\) (\textit{phase 2}) and noise distributions \( \epsilon^i_t \in \mathcal{E}\) (\textit{phase 3}), all of which are derived from the provided original time-series using the selected methods per phase. We now describe the methodology for obtaining the optimal configuration of the TSCM that best fits the original data through ACT.


\subsection{Adversarial Causal Tuning (ACT) \label{subsec:adversarial_causal_tuning}}

\begin{figure}[tbp]
    \centering
    \includegraphics[width=\textwidth]{images/figures/pipeline_bbb_sp2.png}
    \caption{%
        Inspired by AutoML practices and the adversarial learning paradigm of GANs, TCS leverages C2ST as discriminators to assess the quantitated temporal causal models. Each causal model generates synthetic data, which are discriminated against the real data by a robust AutoML pipeline of C2ST models. TCS returns the causal model with the \textit{minimum} discrimination score (\(\mathrm{AUC}_D\)) against the \textit{maximum} performance discriminator.
    }
    \label{fig:pipeline-part-b}
\end{figure}

We leverage on the multiple available options per phase during the TSCM generation, to create a search space \(C\) for the optimal configuration of TCS. At its highest capacity, this search space consists of all possible unique combination of methods (adjusting also for different hyperparameter settings) for all three phases, resulting in the creation of thousands of candidate temporal causal models. To adjust for running efficiency, we restrict the possible simulation configurations to orders of magnitude of tens. To identify the causal model that best describes the original data, each created instance must be assessed based on the similarity of its sampled data empirical distribution to the empirical distribution of the original data. 

Recall that concerning the use of complex parameterized evaluation methods such as C2STs, hyperparameter configuration plays a significant role, especially in case of high-dimensional data with complex functional dependencies (e.g., Figure \ref{fig:tsne}). In general, different metrics may prove to have independent behavior, especially in high dimensions, as shown in \citet{goudet2018learning}. Inspired by the AutoML \citep{hutter2019automated} domain, a configurable discriminator search space is created, with a grid-based search performed to find the C2ST that bests discriminates between generated and original data.

The classifiers that we consider as possible C2ST discriminators include the \textit{Support Vector Classifier (SVC)} \citep{hearst1998support} and the \textit{Long Short-Term Memory-based Classifier (LSTMC)} \citep{hochreiter1997long}, both optimized though hyperparameter tuning \citep{hutter2019automated}, in order to obtain the discriminator that yields the highest \(\mathrm{AUC}_D\). 

By running this process for each candidate causal model, we thus avoid underfit discrimination. This approach highly resembles a discreet, non-differentiable version of adversarial learning in GANs \citep{goodfellow2014generative}, as both the temporal SCM and the discriminator are optimized in parallel while competing with each other. The C2ST problem is deduced to a Min-max selection scheme as follows: Define a set of classifiers \(D\), each with a set of configurations \( C_d = \left\{ c_{d,1}, c_{d,2}, \ldots, c_{d,n} \right\} \). For each classifier \(d \in D\) and configuration \(c \in C_d\), we compute the discriminative score \(d_c\) of the classifier. From these configurations, the maximum discriminative score is selected across the set of lowest scores. That is, optimize with Min-max selection as \[\displaystyle \min_{c \in C} \max_{d \in D} d_c \] The maximization step can be interpreted as \textit{"For each TSCM configuration \(c\), obtain the highest discriminative score across the classifiers"}, while the minimization step as \textit{"Among all TSCM configurations, select the configuration with the lowest maximum score, which depicts the least discriminative, yet optimal case"}. 

In the end, TCS returns the configuration of the optimal temporal causal model, along with the discrimination score and the optimized configuration of the discriminator. This is shown in Figure \ref{fig:pipeline-part-b}, and in pseudocode form in Algorithm \ref{alg:act}.  The considered hyperparameter search spaces for the classifiers (adversarial space) are shown in Table \ref{tab:search-spaces}.

\begin{algorithm}[t]
\caption{Adversarial Causal Tuning (ACT)}
\label{alg:act}
\begin{algorithmic}[1]
\renewcommand{\algorithmicrequire}{\textbf{Input:}}
\renewcommand{\algorithmicensure}{\textbf{Output:}}
\REQUIRE Candidate TCS configurations $\mathcal{S} = \{s_1, \dots, s_m\}$; real test set $\mathcal{D}$; discriminator space $D$; discriminator configuration grids $\{C_d\}_{d \in D}$; significance level $\alpha$; number of permutations $n_{perm}$
\ENSURE Optimal TCS configuration $s_{\text{best}}$ and discrimination score $\mathrm{AUC}_{\text{best}}$
\STATE $\mathcal{R} \gets \emptyset$ \COMMENT{Store best discriminator result for each TCS configuration}
\FOR{each configuration $s \in \mathcal{S}$}
    \STATE $\mathcal{D}'_s \gets \text{SIMULATE}(s)$ \COMMENT{Generate dataset via TCS}
    \STATE $\mathrm{AUC}^s_{\text{best}} \gets -\infty$, $\text{Disc}^s_{\text{best}} \gets \text{None}$, $\text{Cfg}^s_{\text{best}} \gets \text{None}$
    \FOR{each discriminator $d \in D$}
        \FOR{each configuration $c \in C_d$}
            \STATE $\mathrm{AUC}^{c}_{d} \gets \text{TRAIN\_EVAL}(d, c, \mathcal{D}, \mathcal{D}'_s)$
            \IF{$\mathrm{AUC}^{c}_{d} > \mathrm{AUC}^s_{\text{best}}$}
                \STATE $\mathrm{AUC}^s_{\text{best}} \gets \mathrm{AUC}^{c}_{d}$
                \STATE $\text{Disc}^s_{\text{best}} \gets d$
                \STATE $\text{Cfg}^s_{\text{best}} \gets c$
            \ENDIF
        \ENDFOR
    \ENDFOR
    \STATE $\mathcal{R} \gets \mathcal{R} \cup \{(s, \mathrm{AUC}^s_{\text{best}}, \text{Disc}^s_{\text{best}}, \text{Cfg}^s_{\text{best}})\}$
\ENDFOR
\STATE Select configuration $s_o \in \mathcal{R}$ with the lowest $\mathrm{AUC}^s_{\text{best}}$
\STATE Retrieve associated $\text{Disc}_o, \text{Cfg}_o, \mathrm{AUC}_o$
\STATE $\mathcal{E} \gets \{s_o\}$ \COMMENT{Set of statistically equivalent configurations}
\FOR{each configuration $s \in \mathcal{R} \setminus \{s_o\}$}
    \STATE Obtain $probs_o, probs_s, \mathrm{AUC}_o, \mathrm{AUC}_s$ from their discriminators
    \STATE $p \gets \text{PERM\_TEST}(y_{\text{test}}, probs_o, probs_s, \mathrm{AUC}_o, \mathrm{AUC}_s, \alpha, n_{perm})$
    \IF{$p \ge \alpha$}
        \STATE $\mathcal{E} \gets \mathcal{E} \cup \{s\}$ \COMMENT{Add statistically equivalent configuration}
    \ENDIF
\ENDFOR
\STATE $s_{\text{best}} \gets$ configuration in $\mathcal{E}$ with fewest edges in its causal graph
\RETURN $s_{\text{best}}, \mathrm{AUC}^{s_{\text{best}}}_{\text{best}}$
\end{algorithmic}
\end{algorithm}


\subsubsection{Sparsity Penalty}

A drawback of the proposed method is that it may occasionally favor dense graph outputs to more realistic candidate solutions of phase 1, being statistically beneficial during the next phases. To avoid this, we adopt a \textit{sparsity penalty strategy} following \citet{biza2022oct}. Specifically, a permutation-based statistical hypothesis test is introduced on whether two classifiers have statistically equivalent scores. The permutation test assumes as null-hypothesis that two given classifiers have statistically equivalent \(\mathrm{AUC}_D\) scores and uses as its statistic the difference between these two discriminative scores. At its core, the permutation test swaps the per sample predicted probabilities of two classifiers and then re-calculates and compares the permuted outcomes to the ground-truth. If the \(p-\text{value}\) computed by the permutation test is larger than the significance level \(\alpha\), then the comparing \(\mathrm{AUC}_D\)s are considered equivalent. By default, a significance level of \(\alpha=0.05\) and 1000 permutations is used to compute the \(p-\text{value}\). It should be noted that no significant computational overhead is introduced with sparsity penalty, as the only operations performed is the swapping of predictions and the calculation of the permuted \(\mathrm{AUC}_D\) scores. Detailed steps are described in Algorithm \ref{alg:statistical-equivalence-test}. Through this, candidate causal models from the TSCM search space that are statistically equivalent to the best-performing one are filtered. All such candidate models are compared against their sparsity, with the sparsest one being selected, thus avoid the problem of favoring the trivial solution of dense graphs during the search for the optimal causal model.

\begin{algorithm}[t]
\caption{AUC Statistical Equivalence Test (PERM\_TEST)}
\label{alg:statistical-equivalence-test}
\begin{algorithmic}[1]
\renewcommand{\algorithmicrequire}{\textbf{Input:}}
\renewcommand{\algorithmicensure}{\textbf{Output:}}
\REQUIRE Test set labels $y_{\text{test}}$; discriminator test predicted probabilities on the optimal TCSM output $probs_{o}$; discriminator test predicted probabilities on a candidate TCSM output $probs_{i}$; discrimination scores $\mathrm{AUC}_o$ and $\mathrm{AUC}_i$; significance level $\alpha$; number of permutations $n_{perm}$
\ENSURE \textit{True} or \textit{False} for statistical equivalence
\STATE $t_{obs} \gets |\mathrm{AUC}_{o} - \mathrm{AUC}_{i}|$
\STATE $ctr \gets 0$
\FOR{$p = 1$ to $n_{perm}$}
    \STATE $w^{p}_{l} \gets \text{SAMPLE\_PERMUTATION\_INDICES()}$ \COMMENT{Sample indices to swap uniformly}
    \STATE $(probs^{o}_{w}, probs^{i}_{w}) \gets \text{SWAP}(probs_{o}, probs_{i}, w^{p}_{l})$ \COMMENT{Permute sampled indices}
    \STATE $\mathrm{AUC}_{o}^{w} \gets \text{EVALUATE}(y_{\text{test}}, probs^{o}_{w})$
    \STATE $\mathrm{AUC}_{i}^{w} \gets \text{EVALUATE}(y_{\text{test}}, probs^{i}_{w})$
    \STATE $t_{p} \gets |\mathrm{AUC}_{o}^{w} - \mathrm{AUC}_{i}^{w}|$
    \IF{$t_{p} > t_{obs}$}
        \STATE $ctr \gets ctr + 1$
    \ENDIF
\ENDFOR
\STATE $pvalue \gets \frac{ctr}{n_{perm}}$
\IF{$pvalue \ge \alpha$}
    \STATE \textbf{return} True
\ELSE
    \STATE \textbf{return} False
\ENDIF
\end{algorithmic}
\end{algorithm}


\subsection{Evaluation of Simulation Quality} \label{sec:sim-results}

In this section, the capabilities of the proposed \textit{ACT} mechanism are evaluated across two main axes: (i) the ability to accurately recover causal structures from temporal data and (ii) the capacity to simulate realistic temporal dynamics across synthetic and real-world cases. These evaluation experiments aim to assess the robustness of ACT in causal discovery and its competitiveness in temporal data generation against both causal\footnote{Although it has been discussed that CausalTime is not based on a causal model, it is placed in this category for our comparison due to its context.} and non-causal baselines.

The main evaluation metrics for these experiments are: (i) the \textit{Structural Hamming Distance (SHD)} \citep{tsamardinos2006max} for assessing causal structure discovery, computed betweeen the estimated and ground-truth graphs. SHD quantifies the number of graph operations (edge additions, deletions, or reversals) required to match the true graph, where a lower value indicates higher structural accuracy. A reversal accounts for two operations, while a deletion or addition for one. This metric is used to evaluate TCS in Figure \ref{fig:sparsity_penalty}. (ii) \textit{Adjacency-based AUC}, notated \(\mathrm{AUC}_{adj}\). Following \citet{cheng2024causaltime} and \citet{stein2024embracing}, we compute an AUC score based on adjacency matrices of predicted and ground-truth DAGs. The predicted adjacency probabilities are thresholded to \(\tau=0.05\) and compared against the true structure; both matrices are flattened before computing their AUC. This score evaluates the ML efficacy of learned causal structures and is distinct from the sample-level discrimination score \(\mathrm{AUC}_D\) introduced previously.

\subsubsection{Evaluation of the Estimated Causal Structure} \label{subsubsec:sparsity_experiment}

To assess the ability of ACT to correctly infer causal relationships, we generate synthetic datasets with varying levels of complexity. Each dataset consists of ten (10) variables, modeled as an additive noise TSCM with non-linear functional dependencies and Gaussian noise. The underlying causal graphs differ in edge density, ranging from three (3) to thirty (30) edges, thus simulating increasing structural difficulty. These datasets are generated as elaborated in Section \ref{sec:data-synthetic}. Two experimental scenarios are considered: (i) \textit{Fully Connected Candidate Graphs} and \textit{Oracle Ground Truth Candidate Graphs}. 

The first setup examines whether ACT avoids the trivial fully-connected causal structure when such a configuration is included in the search space. Nineteen (19) synthetic datasets are analyzed \textit{(C1)}, each processed by ACT both \textit{with} and \textit{without} the sparsity penalty term. Our results indicate that the sparsity-regularized runs successfully avoided the trivial fully connected solution in all 19 cases, while 8 out of 19 non-regularized runs incorrectly selected it, especially in the case of denser graphs (more than 10 edges). This demonstrates the critical role of sparsity regularization in guiding ACT toward parsimonious causal graphs.

In the second setup, the true causal graph is included as a candidate solution (oracle) in Phase 1, to test whether ACT is able to recover it correctly. With sparsity penalty enabled, ACT successfully identifies the correct causal graph in 16 out of 19 cases (\(\text{SHD}=0\)). Without the sparsity penalty, the true structure is recovered in only 8 out of 19 cases. However, even in mismatched cases, the estimated graphs exhibit low SHD scores, indicating that our method is able to recover solutions that are close to the ground truth. 

\begin{figure*}[t]
    \centering
    \includegraphics[width=\textwidth]{images/figures/sparsity_penalty_random_v3.png}
    (a) \hspace{6.5cm} (b)
    \hfill
    \caption{\textbf{(a)}: Parallel calls of TCS with (w/) and without (w/o) the sparsity penalty, while providing a fully-connected lagged causal graph (referred as \textit{dense graph}) in the \(1^{st}\) phase outputs. The selected causal graphs are compared to the ground truth through SHD, where low SHD scores mean the graphs resemble each other. Calls with sparsity penalty successfully avoid the fully connected graph as a candidate solution at all cases. On the other hand, calls without sparsity penalty end up selecting the fully connected graph 8 out of 19 times, especially for denser cases with 11 or more edges in the ground truth. \textbf{(b)}: Parallel calls of TCS with and without the sparsity penalty, while providing the ground-truth as an oracle lagged causal graph in the \(1^{st}\) phase outputs. Similarity measured through SHD, as in (a). In the vast majority of 16 out of 19 cases, calls with the sparsity penalty successfully chose the oracle graph as the optimal solution. In contrast, calls without the sparsity penalty were able to identify the oracle graph only in 7 out of 19 cases. The results in both (a) and (b) establish the sparsity penalty as a necessary part of TCS.}
    \label{fig:sparsity_penalty}
\end{figure*}

\subsubsection{Comparison to the State-of-the-Art} \label{subsubsec:sim-comparison}

Next, we evaluate the ability of ACT to generate realistic temporal data perform a comparison against both causal and non-causal simulation approaches. The considered methods are (i) \textit{CausalTime} \citep{cheng2024causaltime}, a predictive-modeling-based simulator that captures feature importance but does not perform explicit causal discovery, (ii) \textit{CPAR} \citep{zhang2022sequential}, a conditional probabilistic auto-regressive generative model implemented within the \textit{SDV} framework \citep{patki2016synthetic} and (iii) \textit{TimeVAE} \citep{desai2021timevae}, a variational autoencoder model for time-series generation, implemented via the \textit{SynthCity} library \citep{qian2023synthcity}.

All considered methods are evaluated on both synthetic and real-world datasets, created in accordance with Section \ref{sec:data-training}. For each dataset, 2000-sample sequences are drawn using a block bootstrap approach \citep{gonccalves2011discussion,hardle2003bootstrap}, repeated 10 times in order to compute standard errors. Additionally, the preprocessing protocol of \citep{cheng2024causaltime} is adopted, including linear interpolation for missing values. It should be once more noted that a comparison is inherently unfair, as causal simulation faces a much complex task: ACT generates complete Temporal Structural Causal Models (TSCMs), while CausalTime and other baselines operate at the correlation level without any causal interpretability.

\begin{table}[h!]
\centering
\caption{Benchmarks against non-causal simulation: Comparison of TCS/ACT against causal (CausalTime) and non-causal (CPAR, TimeVAE) simulation methods. Despite solving a more complex causal task, ACT achieves comparable discrimination scores (\(\mathrm{AUC}_D\)) across datasets, often outperforming CausalTime. Lower \(\mathrm{AUC}_D\) values (in \textcolor{red}{red}) indicate more realistic simulations.}
\vspace{8pt}
\setlength{\tabcolsep}{4pt}
\begin{tabular}{lcccc}
\toprule
\multirow{2}{*}{\textbf{Dataset}} & \multicolumn{4}{c}{\textbf{Method}} \\
\cmidrule(lr){2-5}
 & \textbf{TCS (ACT)} & \textbf{CausalTime} & \textbf{CPAR} & \textbf{TimeVAE} \\
\midrule
WTH & $0.85 {\scriptscriptstyle{\pm 0.01}}$ & $0.80 {\scriptscriptstyle{\pm 0.00}}$ & $0.99 {\scriptscriptstyle{\pm 0.00}}$ & $0.98 {\scriptscriptstyle{\pm 0.00}}$ \\
AirQualityUCI & $0.99 {\scriptscriptstyle{\pm 0.01}}$ & $0.83 {\scriptscriptstyle{\pm 0.00}}$ & $0.99 {\scriptscriptstyle{\pm 0.00}}$ & $0.95 {\scriptscriptstyle{\pm 0.00}}$ \\
AirQualityMS & $0.97 {\scriptscriptstyle{\pm 0.00}}$ & $0.84 {\scriptscriptstyle{\pm 0.00}}$ & $0.98 {\scriptscriptstyle{\pm 0.00}}$ & $0.99 {\scriptscriptstyle{\pm 0.00}}$ \\
ETTh1 & $0.92 {\scriptscriptstyle{\pm 0.00}}$ & $0.80 {\scriptscriptstyle{\pm 0.00}}$ & $0.95 {\scriptscriptstyle{\pm 0.00}}$ & $0.91 {\scriptscriptstyle{\pm 0.00}}$ \\
ETTm1 & $0.93 {\scriptscriptstyle{\pm 0.00}}$ & $0.82 {\scriptscriptstyle{\pm 0.00}}$ & $0.96 {\scriptscriptstyle{\pm 0.00}}$ & $0.91 {\scriptscriptstyle{\pm 0.00}}$ \\
Bike Usage & $0.88 {\scriptscriptstyle{\pm 0.01}}$ & $0.77 {\scriptscriptstyle{\pm 0.00}}$ & $0.90 {\scriptscriptstyle{\pm 0.00}}$ & $0.83 {\scriptscriptstyle{\pm 0.00}}$ \\
Outdoors & $0.99 {\scriptscriptstyle{\pm 0.00}}$ & $0.83 {\scriptscriptstyle{\pm 0.00}}$ & $0.99 {\scriptscriptstyle{\pm 0.00}}$ & $0.81 {\scriptscriptstyle{\pm 0.00}}$ \\
\midrule
$f$MRI & \textcolor{red}{\textbf{0.71}} ${\scriptscriptstyle{\pm 0.01}}$ & $0.78 {\scriptscriptstyle{\pm 0.01}}$ & \textcolor{red}{\textbf{0.74}} ${\scriptscriptstyle{\pm 0.02}}$ & $0.75 {\scriptscriptstyle{\pm 0.00}}$ \\
%Finance & $0.99 {\scriptscriptstyle{\pm 0.00}}$ & $0.78 {\scriptscriptstyle{\pm 0.00}}$ & $0.99 {\scriptscriptstyle{\pm 0.00}}$ & $0.86 {\scriptscriptstyle{\pm 0.00}}$ \\
C1 & \textcolor{red}{\textbf{0.57}} ${\scriptscriptstyle{\pm 0.00}}$ & $0.77 {\scriptscriptstyle{\pm 0.00}}$ & \textcolor{red}{\textbf{0.63}} ${\scriptscriptstyle{\pm 0.01}}$ & $0.81 {\scriptscriptstyle{\pm 0.00}}$ \\
\bottomrule
\end{tabular}
\label{tab:non-causal}
\end{table}

Despite addressing the challenging task of causal data generation, ACT achieves \(\mathrm{AUC}_D\) values on par with or better than purely statistical generators, especially on synthetic and semi-synthetic datasets. While non-causal models such as CPAR and TimeVAE focus solely on reproducing correlations, ACT is unique in its ability to produce a complete TSCM. As such, although causal data generation remains a challenging task, our proposed framework remains competitive while being the only approach that returns the estimated TSCM from data. 

A shortcoming of using the Min-max search scheme of ACT, together with the sparsity penalty when attempting to generate hundreds of thousands of instances for training an LCM model is the significant computational overhead. For instance, considering three options for causal discovery, five for functional relationship estimation and two for noise estimator (not taking into account any configurable hyperparameters of each method), this would result in generating a search space of 30 configurations, from which one would select a single data instance. This greatly reduces the amount of training pairs one is able to generate, thus deeming it inefficient for our end scope regarding training pair generation for LCMs. It remains however a significant contribution of this work towards realistic causal data generation itself, as well as for curating datasets for benchmarking causal discovery algorithms. 

\section{Generation of Interventional Samples} \label{sec:data-interventions}

So far our discussion concerned generation of temporal SCMs and observational data from their causal structure. As introduced methodologies (Sections \ref{sec:data-synthetic} and \ref{sec:data-simulated}) output causal models. From these causal models, observational data are then generated. The final pillar of data generation concerns the creation of interventional samples. As elaborated in Chapter \ref{chap:problem-formulation}, an intervention \( \text{do}(V^i_t) = x_t\) corresponds to replacing the functional dependency of \(V^i_t\) in the time-lagged SCM at timestep \(t\) with \( V^i_t := x_t\), in the case of a hard intervention and \( V^i_t := f(\text{Pa}(V^i_t), \epsilon_{V^i}) + x_t \) in the case of a soft intervention. The general method for performing anestral sampling in the TSCM to obtain interventions is illustrated in Algorithm \ref{alg:scm-interventional-sampling}.  

\begin{algorithm}[h]
\caption{Temporal SCM Interventional Sampling (INTERVENTION)}
\label{alg:scm-interventional-sampling}
\begin{algorithmic}[1]
\renewcommand{\algorithmicrequire}{\textbf{Input:}}
\renewcommand{\algorithmicensure}{\textbf{Output:}}
\REQUIRE Temporal Causal Graph $\mathcal{G}$, number of timesteps $T$, max lag $\ell_{\max}$, number of warmup steps $W$, intervention probability $p_{\text{int}} \in [0,1]$, variable-wise value ranges $\mathcal{R} = \{[v_i^{\min}, v_i^{\max}]\}_{i=1}^N$
\ENSURE Interventional time-series sample $\{\mathbf{X}_t\}_{t=1}^{T}$, intervention mask $\mathbf{B} \in \{0,1\}^{T \times N}$
\STATE Initialize $X^i_0$ for all $i \in \{1, \dots, N\}$ with random noise
\STATE Initialize intervention mask $\mathbf{B} \gets \mathbf{0} \in \{0,1\}^{T \times N}$
\FOR{$t = W + \ell_{\max} + 1$ to $T$}
    \FOR{each variable $X^i_t$}
        \IF{Bernoulli$(p_{\text{int}}) = 1$}
            \STATE Sample intervention value $x^i_{\text{int}} \sim \mathcal{U}(v_i^{\min}, v_i^{\max})$
            \STATE Set $X^i_t \gets x^i_{\text{int}}$
            \STATE Set $B_{t,i} \gets 1$
        \ELSE
            \STATE Determine lagged parents:
            \STATE \hspace{1em} $\text{Pa}(X^i_t) \gets \{X^j_{t-k} \mid (X^j, X^i) \in \mathcal{G},\; 1 \leq k \leq \ell_{\max}\}$
            \STATE Compute $X^i_t \gets f_i(\text{Pa}(X^i_t)) + \epsilon^i_t$
        \ENDIF
    \ENDFOR
\ENDFOR
\RETURN $\{\mathbf{X}_t\}_{W+\ell_{\max}}^{T}$, $\mathbf{B}$
\end{algorithmic}
\end{algorithm}

In this work, data collections containing interventions are performed by hard interventions, uniformly sampled from the observational domain of each variable, i.e. \( v^i \sim \mathcal{U}(V_i^{\min}, V_i^{\max})\). Such sampling is performed at each generated timestep and thus accounts for multiple interventional targets. The purpose of the above is two-fold: First, it ensures that interventions remain realistic and within the natural range of the variables, avoiding extreme or implausible values that could distort the dynamics or violate the underlying causal constraints. Second, it promotes diversity and coverage in the interventional data themselves, allowing the model to observe and learn causal effects across the full suport of each variable's distribution. 

Each of these interventional samples is also paired with a \textit{binary intervention mask}, which indicates which variables were intervened on at each timestep. Formally, the mask is \( \mathbf{B} \in \{0,1\}^{T \times N} \), where \( B_{t,i} = 1 \) if variable \( X^i \) was intervened on at timestep \(t\), and \(B_{t,i} = 0\) otherwise. This formulation accounts for both single as well as multiple interventional targets. This does not only serve as a way to test performance of LCMs when training with interventional data as well, but also provide important information for a possible meta-analysis of these data, as an important component of the generation pipeline. The way interventional samples along with the intervention mask are handled by LCMs in particular is explained in Chapter \ref{chap:architecture}.

\section{Curating Training \& Evaluation Data Pairs} \label{sec:data-training}

In this section we provide an overview of the data curation process used to generate the data pairs used for training and evaluation of our LCMs. Three different types of datasets are considered: synthetic (following Section \ref{sec:data-synthetic}), simulated (following Section \ref{sec:data-simulated}) and semi-synthetic. Sections \ref{subsec:data-training-semisynth}, \ref{subsec:data-training-synth} and \ref{subsec:data-training-sim} provide a detailed description of each category of data corpora considered.

\subsection{Semi-synthetic} \label{subsec:data-training-semisynth}

For inference and evaluation purposes, we additionally consider \textit{semi-synthetic} data collections, which combine mechanistic or physically motivated processes, constructed using either domain knowledge or known physical models. Such datasets provide an intermediate bridge between fully synthetic causal structures (where ground truth is known by construction) and real-world observational data (where causal ground truth is unknown or uncertain). 

The \textit{\(f\)MRI} datasets are based on the collection introduced by \citet{smith2011network}, simulating blood-oxygen-level dependent (BOLD) responses across multiple brain regions. Each dataset models the neural activation time-series of a set of interconnected brain regions, with inter-regional causal connectivity defined according to a ground-truth directed network and the hemodynamic response modeled by the non-linear balloon model of \citet{buxton1998dynamics}. This results in temporally smooth, non-linearly coupled time-series that realistically emulate the noise and temporal dependencies observed in empirical \(f\)MRI recordings.  

From the full available collection of 27 datasets with 5, 10 and 15 variables, with sample lengths ranging from 50 to 5000 time steps, we restrict attention to datasets containing at least 500 observations, and exclude the 15-dimensional case to align with our input dimension constraints. This yields a final subset of 26 datasts, denoted as the \textit{\(f\)MRI} collection. The \textit{\(f\)MRI5} collection represents a smaller subset including only 5-dimensional networks. 

To complement the limited number of \(f\)MRI datasets and further assess the ability of our LCMs to capture complex, nonlinear dynamics, we also include semi-synthetic data generated from the Kuramoto model of coupled phase oscillators \citep{kuramoto1975self}. Briefly, the Kuramoto model describes the time evolution of oscillator phases \(\theta_i(t)\) as

\begin{equation}
\frac{d\theta_i}{dt} = \omega_i + \frac{K}{N} \sum_{j=1}^{N} a_{ij} \sin(\theta_j - \theta_i),
\end{equation}

where \(\omega_i\) represents the intrinsic frequency of oscillator \(i\), \(K\) controls the coupling strength, and \(a_{ij}\) encodes the adjacency (causal) structure of interactions among oscillators. The implementation from \citet{lowe2022amortized} is adopted, generating both 5-variable and 10-variable configurations with 1000 realizations each and a fixed maximum lag of 1. 

For all considered semi-synthetic datasets, the known causal connectivity is provided in the form of directed adjacency matrices, which we convert into the lagged adjacency tensor representation described in Section \ref{sec:problem-formulation} and Chapter \ref{chap:architecture} to enable correct comparison of the ground truth with the output of our LCMs. An overview of all semi-synthetic dataset collections is presented in Table \ref{tab:semi-synthetic-datasets}.

\begin{table}[!t]
\centering
\small
\caption{Overview of semi-synthetic time-series dataset collections.}
\vspace{5pt}
\begin{tabular}{|c|c|c|c|c|c|}
\hline
\textbf{Collection} & \textbf{Datasets} & \textbf{Vars} & \textbf{Timesteps} & \textbf{Max Lag} & \textbf{Rels.} \\
\hline
\(f\)MRI5 & 21 & 5 & 1200--5000 & 1 & Non-linear \\
\(f\)MRI & 26 & 5--10 & 1200--5000 & 1 & Non-linear \\
\(\text{Kuramoto}5\) & 1000 & 5 & 500 & 1 & Non-linear \\
\(\text{Kuramoto}10\) & 1000 & 10 & 500 & 1 & Non-linear \\
\hline
\end{tabular}
\label{tab:semi-synthetic-datasets}
\end{table}

\subsection{Synthetic} \label{subsec:data-training-synth}

In addition to the proposed synthetic generation pipeline (Section \ref{sec:data-synthetic}), the synthetic data generator proposed by \citet{stein2024embracing} is also considered as an evaluation test set. Their formulation is based on what they refer to as a \textit{time-invariant SCM with additive noise} (independent Gaussian noise) terms and time-invariant functional mechanisms (thus enforcing causal stationarity), where functional relationships are determined by the lagged adjacency tensor \(\mathbb{A}\) and remain fixed across time. Their generation protocol begins by sampling a random lagged adjacency tensor \(\mathbb{A}\) of predefined dimensionality. It then randomly selects non-zero elements of \(\mathbb{A}\) to generate a single training example, based on a \textit{link threshold parameter} \(\rho \in (0,1)\), which determines the probability \(1 - \rho\) of including a causal edge in the adjacency tensor. A higher threshold would thus enforce sparser graphs. Each non-zero element is assigned a functional dependency \(f_{ij}^{\ell}\) drawn from a predefined collection of functional families, including the non-linear transformations shown in Table \ref{tab:stein-functions} while linear cases are created with a \textit{Vector Autoregressive Model (VAR)} form. From there, observational samples are generated.

As mentioned in Section \ref{sec:data-challenge}, the authors attempt to guarantee stability of generated data samples by forcing boundness of the generated samples into a predefined interval. We found that the non-linear functions \(\text{clamp}_{-0.5,0.5}(x)\), \(x^2\) and \(\frac{1}{x}\) contribute negatively to the stability and computational efficiency of the generator and were thus not considered. To populate each non-zero entry, the authors sample unifomly from a pre-defined range of causal strength values, while various variances are selected for the independent noise terms. Min-max normalization is also applied just like in our generation mechanism.

\begin{table}[ht!]
\centering
\caption{Function families used for nonlinear datasets, Linear cases are parameterized as VAR models, while the nonlinear set (NL) is drawn from a pool of transformations, with one sampled per edge.}
\label{tab:stein-functions}
\begin{tabular}{ll}
\toprule
\textbf{Family} & \textbf{Functional forms} \\
\midrule
Linear & $f(x) = a x + b$ \\
Nonlinear (basic) & $e^x, \; \sin(x), \; \cos(x), \; |x|, \; x$ \\
Nonlinear (bounded/activations) & $\sigma(x), \; \mathrm{ReLU}(x), \; \log(\sigma(x))$ \\
\bottomrule
\end{tabular}
\end{table}

%\begin{table}[ht!]
%\centering
%\caption{Synthetic datasets adapted from the \citet{stein2024embracing} generator to curate the \texttt{S\_Joint} collection. 
%All datasets use additive Gaussian noise ($\sigma^2=0.5$), link threshold $\tau=0.85$, Min-max normalization, and are padded to 5 variables and $\ell_{\max}=3$. }
%\label{tab:stein-datasets}
%\begin{tabular}{llll}
%\toprule
%\textbf{Variables} & \textbf{Max lag} & \textbf{Function class} & \textbf{Coeff. range} \\
%\midrule
%3   & 1--3 & Linear (VAR)   & [0.3, 0.5] \\
%3   & 1--3 & Non-linear (Table \ref{tab:stein-functions}) & [0.3, 0.5] \\
%3   & 1--3 & Linear (VAR) & [0.3, 0.5] \\
%5   & 1--3 & Non-linear (Table \ref{tab:stein-functions}) & [0.3, 0.5] \\
%%7   & 1--3 & same as above & [0.3, 0.5] \\
%%9   & 1--3 & same as above & [0.3, 0.5] \\
%%10  & 1--3 & same as above & [0.3, 0.5] \\
%%12  & 1--3 & same as above & [0.3, 0.5] \\
%\bottomrule
%\end{tabular}
%\end{table}

Using the above generator, we curated a set of dataset collections spanning \(3-5\) variables, with \(\ell_{\max}=3\) and both linear (L) and non-linear (NL) functional forms. These datasets are aggregated into the \texttt{S\_Joint} corpus, which includes \(82,000\) training, \(9,000\) validation, and \(9,000\) test samples. This collection serves as in-distribution, holdout datasets for ablation and generalization studies.

Beyond these external generators, the \texttt{Synth\_230k} corpus is introduced using the synthetic generation pipeline from Section \ref{sec:data-synthetic}, which is used to train large-scale LCMs. The collection extends the diversity of created samples, supporting up to 12 variables and three lags, as well as and ensuring causal stationarity by construction, and consisting of an 80\% training, 10\% validation and 10\% test set as before. Importantly, it surpasses prior training set fforts by more than twofold in total instance count.

\begin{table}[ht!]
\centering
\caption{Overview of synthetic time-series dataset collections used in this work. L denotes linear and NL non-linear. Sizes correspond to 80\% training, 10\% validation, and 10\% test splits.}
\label{tab:synthetic-datasets}
\begin{tabular}{lcccccc}
\toprule
\textbf{Collection} & \textbf{Size} & \textbf{Vars} & \textbf{Timesteps} & \textbf{Max Lag} & \textbf{Rels.} & \textbf{Obs./Interv.} \\
\midrule
\texttt{S\_joint} & 100k & 3--5 & 500 & 1--3 & L, NL & Observational \\
\texttt{Synth\_230k} & 230k & 3--12 & 500 & 1--3 & L, NL & Observational \\
\bottomrule
\end{tabular}
\end{table}


\subsection{Simulated} \label{subsec:data-training-sim}

For the selection of real-data to generate simulated samples, we adopt datasets well-used throughout the deep learning time-series forecasting literature \citep{hahn2023time}, from a variety of real-world domains. These include domains of power, weather, and transportation. The \textit{ETTh1} and \textit{ETTm1} data collections \citep{wu2021autoformer} record hourly and quarter-hourly measurements respectively of electricity transformers, as well as oil and load temperature in a two-year span between July 2016-2018. The \textit{WTH} dataset \citep{jiang2023fecam} consists of 35064 hourly atmospheric measurements, including temperature, humidity, wind speed, direction and pressure starting from January 1st 2010. The \textit{AirQualityUCI} corresponds to 12 hourly measurements of metal oxide chemical sensors in an Italian city. The \textit{bike-usage} dataset contains 52584 hourly wire and infrared sensor measurements of both bike and pedestrian data at the Burke-Gilman Trail in King County, Washington\footnote{Data is \href{https://data.seattle.gov/Transportation/Burke-Gilman-Trail-north-of-NE-70th-St-Bicycle-and/2z5v-ecg8}{publicly available} by the City of Seattle Open Data Portal (last accessed Oct 26, 2024).}, for a total of 5 time-series, while the \textit{Outdoors} dataset contains outdoor BME280 sensor measurements from Nanning, China between February 21st 2016 and November 25th 2016, for a total of 1440 time-steps. Lastly, we include the \textit{Air\_Quality} dataset, consisting of hourly data from 36 monitoring stations across China, as available in the work of \citet{cheng2024causaltime}. An overview is provided in Table \ref{tab:real-datasets}.

\begin{table}[h!]
\centering
\caption{Overview of real time-series datasets used as inputs to the TCS algorithm for generating simulated data pairs.}
\vspace{10pt}
%\resizebox{0.55\textwidth}{!}{%
\begin{tabular}{|c|c|c|c|c|c|}
\hline
\textbf{Dataset} & \textbf{Variables} & \textbf{Timesteps} & \textbf{Granularity} & \textbf{Start Date} & \textbf{Domain} \\ 
\hline
WTH & 12 & 35064 & 1 hour & 01/01/2020 & Weather \\ 
ETTh1 & 7 & 17420 & 1 hour & 07/01/2016 & Power \\ 
ETTm1 & 7 & 69680 & 15 min & 07/01/2016 & Power \\ 
Air\_Quality & 36 & 8760 & 1 hour & - & Weather \\
AirQualityUCI & 12 & 9357 & 1 hour & 03/10/2024 & Weather \\
Bike-usage & 5 & 552584 & 1 hour & 01/01/2014 & Transportation \\
Outdoors & 3 & 1440 & 1 sec & 21/02/2016 & Environmental \\
\hline
\end{tabular}
%}
\label{tab:real-datasets}
\end{table}

For any of these input time-series, we verify for stationarity using the \textit{Augmented Dickey-Fuller test} \citep{dickey1979distribution} (ADF), described in Algorithm \ref{alg:dickey-fuller}. In case non-stationarity is observed, it is normalized using finite differences up to second order. This step assures that data fed into training our LCMs do not possess extreme trends, possibly resulting in unstable training. We illustrate the pseudocode in Algorithm \ref{alg:dickey-fuller}. We follow the python implementation of \texttt{statsmodels} with maximum lag order \( 12 \cdot \left(\frac{T}{100}\right)^{1/4} \) where \(T\) is the number of observed timesteps.

\begin{algorithm}[ht!]
\caption{Dickey-Fuller Test (ADF)} \label{alg:dickey-fuller}
\begin{algorithmic}[1]
\renewcommand{\algorithmicrequire}{\textbf{Input:}}
\renewcommand{\algorithmicensure}{\textbf{Output:}}
\REQUIRE Time-series data $\{y_t\}_{t=1}^{T}$, maximum lag order $p$
\ENSURE Test statistic $\tau$, p-value, decision on unit root
\STATE Compute first differences: $\Delta y_t \gets y_t - y_{t-1}$ for $t = 2, \dots, T$
\STATE Construct regression design matrix with:
    \begin{enumerate}
        \item Regressor $y_{t-1}$
        \item Deterministic terms (constant, trend) if included
        \item Lagged differences $\Delta y_{t-k}$ for $k = 1, \dots, p$
    \end{enumerate}
\STATE Estimate regression by OLS:
    \[
    \Delta y_t = \alpha + \beta t + \gamma y_{t-1} + \sum_{k=1}^p \phi_k \Delta y_{t-k} + \epsilon_t
    \]
\STATE Extract test statistic $\tau \gets \hat{\gamma} / \mathrm{SE}(\hat{\gamma})$
\STATE Compare $\tau$ against Dickey-Fuller critical values
\IF{$\tau < \tau_{\text{critical}}$}
    \STATE Reject null hypothesis (series is stationary)
\ELSE
    \STATE Fail to reject null hypothesis (series has a unit root)
\ENDIF
\RETURN Test statistic $\tau$, p-value, decision
\end{algorithmic}
\end{algorithm}

Ultimately, we introduce a large-scale \textit{simulated} dataset collection, denoted \texttt{sim\_45K}. These datasets are derived from real multivariate time-series sources by simulating causal models under the Temporal Causal-based Simulation (TCS) framework (Subsection \ref{subsec:tcs}). Starting from well-established real-world datasets in the time-series forecasting literature \citep{lai2018modeling, wu2023timesnet}, we generate realistic causal variants using three main data augmentation techniques: (i) \textit{Time-window subsampling:} extracting multiple overlapping or disjoint temporal segments from long sequences to simulate diverse initial conditions and phase shifts; (ii) \textit{Node subset sampling:} selecting variable subsets to emulate local subsystem dynamics and improve structural diversity, and (iii) \textit{Estimator variability:} varying the causal estimation and simulation parameters within ACT to produce multiple realizations of plausible causal graphs from a single base dataset.

This procedure yields over 45,000 high-quality simulated instances that preserve the statistical structure of real data while offering known causal ground truth. Together with the synthetic corpora (\texttt{synth\_230k}), these datasets form the combined large-scale training corpus \texttt{synth\_230k\_sim\_45k}, which is \textit{used for training large-scale LCMs}. The combined dataset totals \textit{more than 275,000 examples}, more than double the size of previous causal time-series foundations. Additionally, the \texttt{AirQualityMS} dataset is introduced as a realistic out-of-distribution benchmark, from which we create \(50\) simulated instances by ACT for further evaluating our trained models.

In addition to purely synthetic and simulated collections, we curated a set of \textit{mixture datasets} designed to empirically investigate the optimal balance between synthetic and realistic (simulated) data during large-scale model training, in line with recent findings by \citet{das2024decoder} as elaborated in Section \ref{sec:data-challenge}. To this end, we curate four mixture datasets with varying ratios of synthetic to simulated data, based on the \texttt{synth\_230k\_sim\_45k} dataset, each consisting of \(50,000\) paired instances. All datasets preserve consistent temporal characteristics, with variable count in the range \([3,12]\), maximum lag \(\ell_{\max}=3\) and 500 time steps. Only data composition is altered. Each collection follows an 80\% / 10\% / 10\% split for training, validation, and test data as in the previous cases, as shown in Table \ref{tab:mixture-datasets}.

\begin{table}[ht!]
\centering
\caption{Mixture dataset subsets curated to explore the effect of varying synthetic-to-simulated ratios on training dynamics and generalization. Each subset contains 50k data pairs.}
\label{tab:mixture-datasets}
\begin{tabular}{lcccc}
\toprule
\textbf{Collection} & \textbf{Size (pairs)} & \textbf{Synthetic (\%)} & \textbf{Simulated (\%)} \\
\midrule
\texttt{mix\_100s\_0sim} & 50k & 100 & 0 \\
\texttt{mix\_80s\_20sim} & 50k & 80 & 20 \\
\texttt{mix\_50s\_50sim} & 50k & 50 & 50 \\
\texttt{mix\_20s\_80sim} & 50k & 20 & 80 \\
\bottomrule
\end{tabular}
\end{table}

These collections are introduced in Chapter \ref{chap:results} to explore the optimal combination of simulated and synthetic data instances for training large-scale LCMs and hether they benefit out-of-distribution performance compared to training on synthetic data alone.

\begin{table}[ht!]
\centering
\caption{
Overview of the large-scale synthetic and simulated dataset collections used for training, ablation, and evaluation of large causal models (LCMs). Each dataset follows a 80\%/10\%/10\% split for training, validation, and test data. \texttt{Synthetic\_230k} forms the primary synthetic corpus; \texttt{Sim\_45K} comprises simulated (real-derived) data for realism evaluation; \texttt{Synth\_230k\_sim\_45k} is their combined mixture used for large-scale training and mixture analysis; and \texttt{S\_Joint} (adapted from \citet{stein2024embracing}) supports ablation studies under varying causal configurations.}
\label{tab:final-train-collections}
\begin{tabular}{lcccc}
\toprule
\textbf{Collection} & \textbf{Type} & \textbf{Instances} & \textbf{Vars} & \textbf{Lags} \\
\midrule
\texttt{Synthetic\_230k} & Synthetic (Section \ref{sec:data-synthetic}) & 230k & 3--12 & 1--3 \\
\texttt{Sim\_45K} & Simulated (Section \ref{sec:data-simulated})& 45k & 3--12 & 1--3 \\
\texttt{Synth\_230k\_sim\_45k} & Mixture & 275k & 3--12 & 1--3 \\
\texttt{S\_Joint} & Synthetic (\citet{stein2024embracing}) & 100k & 3--5 & 1--3 \\
\bottomrule
\end{tabular}
\end{table}

Finally, we also introduce a synthetic corpus combining observational and interventional samples, \texttt{Synth\_40k}, which is introduced for preliminary experiments in Chapter \ref{chap:results} to explore the effect of interventions on the performance of LCMs. As the name suggests, it consists of forty thousand synthetic instances, of the same splits as the previous collections, up to \(5\) variables and \(\ell_{\max}=3\). Each instance consists not only of the ground truth lagged causal graph and the observed time series, but also of the intervened samples, created as in Section \ref{sec:data-interventions}.


\subsection{Data Shardification} \label{subsec:shardification}

Given the large scale of the combined dataset (totaling more than 275,000 paired instances), it becomes impractical to load the entire collection into memory simultaneously during training. To address this limitation, we adopt a \textit{shardification} approach, in which the dataset is partitioned into multiple serialized files (\texttt{.pt} shards), each containing a manageable subset of samples. This design allows efficient storage, distribution, and access to data instances without compromising training throughput.

To support seamless access across all shards, a custom data loader is introduced, treating the collection of sharded files as a single, unified dataset. During training, shards are loaded on demand (only the file corresponding to the currently accessed index range is read into memory, while previous shards are released). This loading mechanism substantially reduces memory footprint and enables training on systems with limited resources, while maintaining constant-time indexing behavior for random sample retrieval.

Each shard is dynamically wrapped by the main data loader, which applies standardized preprocessing steps such as normalization, temporal alignment, and fixed-length padding or truncation to \(L_{\max}\). To prevent data leakage across shards, all samples are deeply cloned before returning, guaranteeing memory isolation and reproducible sampling.

At the beginning of each training epoch, the shard order is randomized to promote data diversity and prevent potential overfitting to shard-specific distributions. The result is an efficient, scalable data pipeline that allows LCMs to be trained over hundreds of thousands of simulated and synthetic causal time-series pairs without exceeding hardware memory limits.