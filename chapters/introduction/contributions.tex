\section{Guide to this Thesis}

\begin{figure}[h!]
\centering
\begin{tikzpicture}[
    node distance=1.4cm and 1.8cm,
    every node/.style={draw, minimum width=2.2cm, minimum height=1.0cm, align=center, font=\small},
    solidarrow/.style={->, thick},
    dottedarrow/.style={->, thick, dotted}
]

% Top row
\node (intro) {1. Introduction};
\node (context) [right=of intro] {2. Context \& \\ Problem Formulation};
\node (data) [right=of context] {3. Training Data \\ Generation};

% Bottom row
\node (arch) [below=of intro] {4. Architecture};
\node (train) [right=of arch] {5. Training};
\node (results) [right=of train] {6. Results};
\node (concl) [right=of results] {7. Conclusion};

% Solid arrows
\draw[solidarrow] (intro) -- (context);
\draw[solidarrow] (context) -- (data);
\draw[solidarrow] (data.south) -- ++(0,-0.5) -| (arch.north);
\draw[solidarrow] (arch) -- (train);
\draw[solidarrow] (train) -- (results);
\draw[solidarrow] (results) -- (concl);

% Dotted cross-links (centered to Results)
\draw[dottedarrow] (intro.south) .. controls +(0,-1) and +(0,-0.5) .. (train.west);
\draw[dottedarrow] (context.south) .. controls +(0,-1.0) and +(0,1.0) .. (results.north west);
\draw[dottedarrow] (data.south) .. controls +(0,-1.0) and +(0,1.0) .. (results.north);

\end{tikzpicture}
\caption{Roadmap of this thesis. Solid arrows denote the main sequential flow of chapters. Dotted arrows indicate supporting or cross-referenced dependencies between chapters.}
\label{fig:thesis-roadmap}
\end{figure}

This section provided a high-level overview of our text. Chapter \ref{chap:problem-formulation} serves as an introduction to needed background, mainly in Causality and briefly to Deep Learning. It introduces Structural Causal Models (SCMs), their temporal extensions and known assumptions. Concerning Deep Learning, we provide a short overview on Deep Learning terminology for the uninitiated reader and elaborate on the Transformer architecture, which forms the backbone of our LCM, and on building blocks of neural nets that we utilize later on. In section \ref{sec:problem-formulation}, we formally define the main objective of our work. Chapter \ref{chap:data} details our synthetic data pipeline and the Temporal Causal Simulation (TCS) \& Adversarial Causal Tuning (ACT) frameworks for generating data pairs for training. Chapter \ref{chap:architecture} is dedicated to the architecture of our LCMs and input handling. Chapter \ref{chap:training} discusses training objectives, optimizers, and regularization methods that are essential for training of LCMs. Chapter \ref{chap:results} evaluates trained LCMs against baselines, with ablation and generalization studies. Chapter \ref{chap:conclusion} concludes the text with contributions, limitations, and future directions. We illustrate a roadmap of this thesis in Figure \ref{fig:thesis-roadmap}.